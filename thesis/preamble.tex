%% ============================================================================
%%
%%  Master's thesis
%%
%%  Author: Rune Thorsen
%%
%%  Preamble
%%
%%  Original preamble: Jakob Lysgaard Rørsted (Mosumgaard)
%% ============================================================================

% ~~~~~~~~~~~~~~~~~~~~~~~~~~~~~~~~~~~~~~~~~~~~~~~~~~~~~~~~~~~~~~~~~~~~~~~~~~~~~
% Core
% ~~~~~~~~~~~~~~~~~~~~~~~~~~~~~~~~~~~~~~~~~~~~~~~~~~~~~~~~~~~~~~~~~~~~~~~~~~~~~
% Essentials
\usepackage[utf8]{inputenc}
\usepackage[T1]{fontenc}

% Microtype -- Subliminal refinements towards typographical perfection
\usepackage{microtype}
\microtypesetup{final}
% \microtypesetup{tracking = true}

% Various tools needed in the preamble and by some packages
\usepackage{etoolbox}

% Language (has to be loaded before the fonts..?)
% Load both languages to make two different abstracts
\usepackage[danish,english]{babel}
\renewcommand\danishhyphenmins{22}
\selectlanguage{english}

% Fonts: Linux Libertine (http://www.tug.dk/FontCatalogue/linuxlibertine/)
\usepackage{libertine}            % Linux Libertine as text font
\usepackage{libertinust1math}     % Math support for Linux Libertine
\usepackage[scaled=.95]{newtxtt}  % Pretty teletype in correct size


% ~~~~~~~~~~~~~~~~~~~~~~~~~~~~~~~~~~~~~~~~~~~~~~~~~~~~~~~~~~~~~~~~~~~~~~~~~~~~~
% Page 'n' stuff
% ~~~~~~~~~~~~~~~~~~~~~~~~~~~~~~~~~~~~~~~~~~~~~~~~~~~~~~~~~~~~~~~~~~~~~~~~~~~~~
% Page set-up
\setlrmarginsandblock{3.2cm}{*}{1.5}
\setulmarginsandblock{*}{3.7cm}{1}

% Some memoir tricks
\setlength{\topskip}{1.6\topskip}
\checkandfixthelayout
\sloppybottom
\strictpagechecktrue

% To remove the blank page after the titlepage to save space
% --> Only for this progress report! Normally the blank page should be there!!
\usepackage{atbegshi}


% ~~~~~~~~~~~~~~~~~~~~~~~~~~~~~~~~~~~~~~~~~~~~~~~~~~~~~~~~~~~~~~~~~~~~~~~~~~~~~
% Page and chapter styles
% ~~~~~~~~~~~~~~~~~~~~~~~~~~~~~~~~~~~~~~~~~~~~~~~~~~~~~~~~~~~~~~~~~~~~~~~~~~~~~

% NOTE: Colors added everywhere, to make them easy to change!
\usepackage{xcolor}

% Fonts
\newcommand{\foliofont}{\color{black}\sffamily}
\newcommand{\headerfont}{\color{black}\sffamily}

% Make new pagestyle
\makepagestyle{jrmbsc}
\makeevenhead{jrmbsc}{\foliofont\thepage}    {} {\headerfont\leftmark}
\makeoddhead {jrmbsc}{\headerfont\projecttitle} {} {\foliofont\thepage}
\makeevenfoot{jrmbsc}{}{}{}
\makeoddfoot {jrmbsc}{}{}{}

% Black magic happens below! (define pagestyle)
\makeatletter
\makepsmarks {jrmbsc}{
  % Syntax: \createmark{<division type}{left|right|both marks}{shownumber|nonumber}{prefix}{postfix}
  \createmark{chapter}    {both}  {shownumber} {\@chapapp\ } {\ $\cdot$\ }
  \createmark{section}    {right} {shownumber} {}            { \ }
%  \createmark{subsection} {right} {nonumber}   {}            {}
  \createplainmark{toc}   {both}  {\contentsname}
  \createplainmark{lof}   {both}  {\listfigurename}
  \createplainmark{lot}   {both}  {\listtablename}
  \createplainmark{bib}   {both}  {\bibname}
  \createplainmark{index} {both}  {\indexname}
}
\makeatother
\nouppercaseheads

% Sections with sans-serif
\setsecheadstyle{\Large\bfseries\sffamily\raggedright}
\setsubsecheadstyle{\large\bfseries\sffamily\raggedright}

% Fix of the pagestyle of the chapter-pages
\copypagestyle{chapter}{empty}
\makeoddfoot{chapter}{}{\foliofont\thepage}{}
\makeevenfoot{chapter}{}{\foliofont\thepage}{}

% Chapterstyle (modified from Rasmus Villemoes thesis)
\usepackage{graphicx}
\makechapterstyle{jrmbsc}{% requires graphicx package
  \chapterstyle{default}
  \renewcommand*{\chapnamefont}{%
    \normalfont\LARGE\color{black}\scshape\raggedleft}
  \renewcommand*{\chaptitlefont}{%
    \normalfont\Huge\color{black}\sffamily\raggedleft}
  \renewcommand*{\chapternamenum}{}
  \renewcommand*{\printchapternum}{%
    \makebox[0pt][l]{\hspace{0.4em}
      \resizebox{!}{5ex}{%
        \normalfont\Large\color{black}\thechapter}
    }%
  }%
  \renewcommand*{\afterchapternum}{%
    \par\hspace{1.5cm}\color{black}\hrule\vskip\midchapskip}}

% Use the new pagestyle and chapterstyle
\pagestyle{jrmbsc}
\chapterstyle{jrmbsc}

% Turn on numbering of paragraphs
\setsecnumdepth{paragraph}


% ~~~~~~~~~~~~~~~~~~~~~~~~~~~~~~~~~~~~~~~~~~~~~~~~~~~~~~~~~~~~~~~~~~~~~~~~~~~~~
% Science
% ~~~~~~~~~~~~~~~~~~~~~~~~~~~~~~~~~~~~~~~~~~~~~~~~~~~~~~~~~~~~~~~~~~~~~~~~~~~~~
% Basic math (might already be loaded by the math font package)
\usepackage{amsfonts,amssymb,bm} % God matematik
\usepackage{mathrsfs}
\usepackage{stmaryrd}

% Bad-ass math!
\usepackage{mathtools}
\mathtoolsset{showonlyrefs=false,showmanualtags}
%\mathtoolsset{showonlyrefs=true}  % If only to show numbers on ref'ed eq's

% Units
\usepackage{siunitx}
\sisetup{separate-uncertainty=true}

% Computer-related units
\DeclareSIUnit\byte{B}

% Delimeters
\newcommand{\abs}[1]{\ensuremath{\left\vert #1 \right\vert}\xspace}

% Tal og talmængder
\newcommand{\e}{\mathrm{e}} %Eulers tal
\newcommand{\N}{\mathbb{N}} %De naturlige tal
\newcommand{\Z}{\mathbb{Z}} %Heltallene
\newcommand{\Q}{\mathbb{Q}} %De rationelle tal
\newcommand{\A}{\mathbb{A}} %De algebraiske tal
\newcommand{\R}{\mathbb{R}} %De reelle tal
\newcommand{\I}{\mathbb{I}} %De imaginære tal
\newcommand{\C}{\mathbb{C}} %De komplekse tal
\newcommand{\F}{\mathbb{F}} %Et vilkårligt legeme
\newcommand{\K}{\mathbb{K}} %Endnu et vilkårligt talrum

% Algoritmer
\usepackage{algpseudocode,algorithm}


% ~~~~~~~~~~~~~~~~~~~~~~~~~~~~~~~~~~~~~~~~~~~~~~~~~~~~~~~~~~~~~~~~~~~~~~~~~~~~~
% Floats, captions, footnotes and counters
% ~~~~~~~~~~~~~~~~~~~~~~~~~~~~~~~~~~~~~~~~~~~~~~~~~~~~~~~~~~~~~~~~~~~~~~~~~~~~~
% Include graphics
% \usepackage{graphicx}  % ALREADY IMPORTED

% Specificér counter til equations
\counterwithin{equation}{chapter}

% Skift counter til algoritmer
\makeatletter
\let\c@algorithm\c@equation
\renewcommand{\thealgorithm}{\thechapter.\arabic{equation}}
\makeatother

% Gode tabeller
\usepackage{array,tabularx,booktabs,authblk,hhline}

% Skift counter til figurer
\makeatletter
\let\c@figure\c@equation
\makeatother

% Skift counter til tabeller
\makeatletter
\let\c@table\c@equation
\makeatother

% Subfloats
\newsubfloat{figure}
\subcaptionstyle{\raggedright}

% Use sans-serif for captions (alternative layout: Change width (see below))
\captionnamefont{\sffamily\scshape}
\captiontitlefont{\sffamily\small}

% Width of caption --> Use sf-font instead
% \captionwidth{.8\linewidth}
% \changecaptionwidth

% Trick to automatically end captions with a period
\captiontitlefinal{.}

% Styling of the footnotes (memoir tricks)
\setlength{\footmarkwidth}{-1sp}
\setlength{\footmarksep}{0em}
\footmarkstyle{#1: }

% Cool tables with footnotes (using the same style as define just above)
\usepackage[online]{threeparttable}
\appto\TPTnoteSettings{\footnotesize}


% ~~~~~~~~~~~~~~~~~~~~~~~~~~~~~~~~~~~~~~~~~~~~~~~~~~~~~~~~~~~~~~~~~~~~~~~~~~~~~
% Stuff
% ~~~~~~~~~~~~~~~~~~~~~~~~~~~~~~~~~~~~~~~~~~~~~~~~~~~~~~~~~~~~~~~~~~~~~~~~~~~~~
% Spacing in macros
\usepackage{xspace}

% Debugging
\usepackage{lipsum}
\usepackage[margin,draft]{fixme}
\fxusetheme{color}
% \fxnote (grøn), \fxerror (gul), \fxwarning (orange), \fxfatal (rød)

% Front page and colophon
\usepackage{soul}
\sodef\spread{}{.2em}{.9em plus.4em}{1em plus.1em minus.1em}
\newcommand{\packagename}[1]{\texttt{#1}}

% Things with draft
\usepackage[firstpage]{draftwatermark}
\SetWatermarkText{\sffamily DRAFT}

% Nice itemizations
\usepackage{enumitem}
% \firmlists  % Activate firmlists everywhere?

% Logo
\usepackage{metalogo}
\setlogokern{La}{-0.265em}
\setlogokern{aT}{-0.09em}
\setlogokern{Te}{-0.07em}
\setlogokern{eX}{-0.072em}
\setlogokern{eT}{-0.056em}
\setlogodrop{0.158em}

% Multiple abstracts (dirty hack?)
\newenvironment{multiabstract}[1]
{\renewcommand{\abstractname}{#1}\begin{abstract}}
{\end{abstract}}


% ~~~~~~~~~~~~~~~~~~~~~~~~~~~~~~~~~~~~~~~~~~~~~~~~~~~~~~~~~~~~~~~~~~~~~~~~~~~~~
% Theorems (add if required)
% ~~~~~~~~~~~~~~~~~~~~~~~~~~~~~~~~~~~~~~~~~~~~~~~~~~~~~~~~~~~~~~~~~~~~~~~~~~~~~
\usepackage{xpatch}
\usepackage{courier}

% % Sætninger og beviser (opsætning længere nede)
\usepackage[standard,thmmarks,amsmath]{ntheorem}

% % Definition af sætning, lemma og korollar mm. med fortløbende numerering
\newtheorem{thm}[equation]{Theorem}%[section]
\newtheorem{lem}[equation]{Lemma}
\newtheorem{cor}[equation]{Corollary}
\newtheorem{defi}[equation]{Definition}
\newtheorem{prop}[equation]{Proposition}
%\newtheorem{remark}[algorithm]{Remark}
%\newtheorem{example}[algorithm]{Example}
\newtheorem{prob}[equation]{Problem}
\newtheorem{conj}[equation]{Conjecture}


% ~~~~~~~~~~~~~~~~~~~~~~~~~~~~~~~~~~~~~~~~~~~~~~~~~~~~~~~~~~~~~~~~~~~~~~~~~~~~~
% References
% ~~~~~~~~~~~~~~~~~~~~~~~~~~~~~~~~~~~~~~~~~~~~~~~~~~~~~~~~~~~~~~~~~~~~~~~~~~~~~
% Smart quotations
\usepackage[style=english]{csquotes}

% URL's
\usepackage{url}

% Bibliography
\usepackage[backend=biber,
  style=numeric-comp,     % Citation style as [NUMBER]
  sortcites=true,
  maxcitenames=1,         % Increase/decrease to include more/fewer authors in cites
  maxbibnames=5,          % As above, but in the bibliography
  uniquelist=false,
  uniquename=false,
  doi=false,
  url=false,
  isbn=false,
  eprint=false,
  hyperref=true]{biblatex}

% Actually apply the citation order (because the bibliography is sorted differently)
\assignrefcontextentries[]{*}

% Space in bibliography (change to compress/expand bibliography)
\setlength\bibitemsep{1.3\itemsep}

% Change bib-order
\DeclareNameAlias{sortname}{last-first}

% Load the file with bibliographic information
\addbibresource{bibliography.bib}

% DANISH STUFF: Et al. på dansk
% \DefineBibliographyStrings{danish}{%
%   andothers = {et\addabbrvspace al\adddot}
% }
  
% Referencing packages (needs to be loaded in this order!)
% --> For references, use: \cref{}  or \Cref{} !
\usepackage{refcount}
\usepackage{varioref}
\usepackage[
  unicode=true,
  pdftitle={\projecttitle},
  pdfauthor={Rune N. T. Thorsen},  % Change this name!
  pdfkeywords={},
  bookmarksopen=true,
  pdfdisplaydoctitle=true,
  hypertexnames=false]{hyperref}
\usepackage{cleveref}

% Hyperref setup (from Rasmus Villemoes)
\makeatletter
\@ifpackageloaded{hyperref}{
  \hypersetup{colorlinks=false, pdfborder=0 0 0}
  \addto\extrasenglish{ % What does this do???
    \renewcommand\subsectionautorefname{Subsection}%
    \renewcommand\sectionautorefname{Section}%
    \renewcommand\chapterautorefname{Chapter}%
    \renewcommand\equationautorefname{equation}%
  }
}{}
\makeatother

% % Extra names for environments that cleveref can reference
\crefname{thm}{theorem}{theorems}
\crefname{lem}{lemma}{lemmas}
\crefname{cor}{corollary}{corollaries}
\crefname{defi}{definition}{definitions}
\crefname{prop}{proposition}{propositions}
%\crefname{remark}{remark}{remarks}
%\crefname{example}{example}{examples}
\crefname{prob}{problem}{problems}
\crefname{conj}{conjecture}{conjectures}
%\crefname{algorithm}{algorithm}{algorithms}


% ~~~~~~~~~~~~~~~~~~~~~~~~~~~~~~~~~~~~~~~~~~~~~~~~~~~~~~~~~~~~~~~~~~~~~~~~~~~~~
% Macros
% ~~~~~~~~~~~~~~~~~~~~~~~~~~~~~~~~~~~~~~~~~~~~~~~~~~~~~~~~~~~~~~~~~~~~~~~~~~~~~
% Nice spacing in macros
\usepackage{xspace}

% Seperation in equation
\newcommand\eqsep{\ensuremath{\quad , \quad}}

% Vectors (we are doing physics!)
\renewcommand{\vec}[1]{\ensuremath{\boldsymbol{\mathrm{#1}}}\xspace}

% Nice subscript
\newcommand\var[2]{\ensuremath{#1_{\textup{#2}}\,}\xspace}

% Size of a set
\newcommand{\size}[1]{\ensuremath{\left\vert #1 \right\vert}\xspace}

% Concatenation of strings operator
\newcommand{\concat}[2]{\ensuremath{#1 \mathbin{\vert\vert} #2}\xspace}

% XOR operator
\newcommand{\xor}{\ensuremath{\mathbin{\oplus}}\xspace} % Small
\newcommand{\Xor}{\ensuremath{\mathop{\mathbin{\bigoplus}}}\xspace} % Large (works with \limits)

% Bra-Ket notation
\DeclarePairedDelimiter\bra{\langle}{\rvert}
\DeclarePairedDelimiter\ket{\lvert}{\rangle}

% For multi-line equations (using aligned) with intertext
% This should NOT be used in general!
% Also it is placed here, because it needs to be loaded after amsmath (which is loaded by ntheorem)
\newcommand{\alignedintertext}[1]{
  \noalign{
    \vskip\belowdisplayshortskip
    \vtop{\hsize=\linewidth\noindent#1\par
    \expandafter}
    \expandafter\prevdepth\the\prevdepth
  }
}

% Used for automatic scaling of parentheses using \pmod{}
\renewcommand{\pod}[1]{\allowbreak
   \if@display\mkern18mu\else\mkern8mu\fi\left(#1\right)}

